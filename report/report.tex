\documentclass[12pt,twoside]{report}
\usepackage{amsmath}
\usepackage{amsfonts}

% some definitions for the title page
\newcommand{\reporttitle}{asdfasdf}
\newcommand{\reportauthor}{Amelia Gordafarid Crowther}
\newcommand{\supervisor}{Name of supervisor}
\newcommand{\reporttype}{Type of Report/Thesis}
\newcommand{\degreetype}{Computer Science BEng} 

% load some definitions and default packages
\input{includes}

% load title page
\begin{document}
\input{titlepage}


% page numbering etc.
\pagenumbering{roman}
\clearpage{\pagestyle{empty}\cleardoublepage}
\setcounter{page}{1}
\pagestyle{fancy}

%%%%%%%%%%%%%%%%%%%%%%%%%%%%%%%%%%%%
\begin{abstract}
Your abstract.
\end{abstract}

\cleardoublepage
%%%%%%%%%%%%%%%%%%%%%%%%%%%%%%%%%%%%
\section*{Acknowledgments}
Comment this out if not needed.

\clearpage{\pagestyle{empty}\cleardoublepage}

%%%%%%%%%%%%%%%%%%%%%%%%%%%%%%%%%%%%
%--- table of contents
\fancyhead[RE,LO]{\sffamily {Table of Contents}}
\tableofcontents 


\clearpage{\pagestyle{empty}\cleardoublepage}
\pagenumbering{arabic}
\setcounter{page}{1}
\fancyhead[LE,RO]{\slshape \rightmark}
\fancyhead[LO,RE]{\slshape \leftmark}

%%%%%%%%%%%%%%%%%%%%%%%%%%%%%%%%%%%%
\chapter{Introduction}

\begin{figure}[tb]
\centering
\includegraphics[width = 0.4\hsize]{./figures/imperial}
\caption{Imperial College Logo. It's nice blue, and the font is quite stylish. But you can choose a different one if you don't like it.}
\label{fig:logo}
\end{figure}

Figure~\ref{fig:logo} is an example of a figure. 

\section{Motivation}

The reconstruction of 3-dimensional scenes using RGB-D data has been the subject of a lot of research recently. Due to the wide range of applications in fields such as Robotics and Augmented Reality, it has gained much interest.

However, most existing systems are designed only to reconstruct static geometry. This leads to hard limitations in the adaptability of the system to a changing environment. Worse still, highly deformable or dynamic objects such as a moving person cannot be reconstructed at all by a static system.

There has been a lot of interest in techniques for non-rigid reconstruction, but despite it being a hot research topic, few implementations of proposed systems are widely available.

The aim of this project is to provide such an implementation that can reconstruct a rigid object from a  non-rigid scene in real-time.


%%%%%%%%%%%%%%%%%%%%%%%%%%%%%%%%%%%%
\chapter{Background}

\section{Kinect Camera}

The first and most fundamental element of the system is the sensor used to obtain information about the surface to be reconstructed. The most commonly used sensor for this purpose is the Kinect camera. Originally designed as a commodity level camera for home games consoles, it was available directly to consumers. This means that Kinect cameras are widely and cheaply available to use.

However, they provide high quality information considering their price. This is provided to the system in the form of an ordinary RGB image, augmented with a fourth parameter indicating the perceived depth from the camera at that pixel. Using the available depth information, it is also possible to compute the normal to the surface at each pixel of the image. In some cases, depth information may be missing at certain pixels, producing so-called `holes' in the depth image. Errors such as these are not difficult to overcome but must be handled.


The sensor is capable of producing an RGB-D image to the system in 640x480 resolution at up to 30Hz. This is important, as it determines the amount of information that can be added to the system at each update. Additionally, it places a hard limit on the update rate of the system, which can never exceed the sensor rate.

\section{Rigid Surface Reconstruction}
\subsection{KinectFusion}
\section{Non-rigid Surface Reconstruction}
\subsection{DynamicFusion}

\subsection{VolumeDeform}

\section{SDF-2-SDF}

SDF-2-SDF is a technique for static reconstruction.

precursor to KillingFusion, and designed in a very similar way

operates more explicitly - no notion of point clouds, so data of depth image is abstracted into a SDF. 

creates a canonical SDF by accumulating data from each frame

\subsection{Signed distance field}

A signed distance field (SDF) is a mapping $\phi : \mathbb{R}^3 \rightarrow  \mathbb{R}$ which yields the distance from the surface of an object at a particular point in space. 
Points inside the object correspond to negative values; points outside the object correspond to positive values and points on the surface are mapped to zero.

One of the important properties of a signed distance field is that the magnitude of the gradient of the SDF is equal to one at all points. This is called the level set property and is useful as a regularizer.

additionally, the gradient of the sdf is equivalent to the normal of the surface at the given point.

\subsection{Voxel grid}

the volume to be reconstructed is divided up into a regular grid of voxels. resolution 5-10 millimetres - mention sensor error of kinect camera

a gradient descent scheme is used to iteratively align the current SDF to the canonical sdf. this process is repeated until the magnitude of the maximum update is less than a certain threshold (0.1 millimetres)

$$\psi_{n+1} = \psi_{n} - \alpha E'(\psi)$$

this process is repeated for each vector $\psi$ in the voxel grid across the full deformation field $\Psi$.



\subsection{energy functions}

simplest:

$$ E_{geom} = \frac{1}{2} \sum\limits_{\textrm{all voxels}} (\phi_n(\Psi) - \phi_{global})^2  $$


additionally, the normals can be constrained to be approximately 1. Originally the dot product of the two relevant normals was used

$$ E_{norm} = \sum\limits_{\textrm{all voxels}}(1 - \nabla \phi_n(\Psi) \cdot \nabla\phi_{global}) $$

These two error functions can be combined proportionally using a scaling factor $\alpha_{norm}$. This factor can take on values in the range $[0..1]$, but is usually set to zero to improve performance.

$$E = E_{geom} + \alpha_{norm}E_{norm}$$

$$ E'_{geom} = (\phi_n(\Psi) - \phi_{global}) \nabla \phi_n(\Psi)$$
\subsection{Weight average accumulation}

\begin{align*}
\Phi_{n+1}(\textbf{V}) &= \frac{W_n(\textbf{V})\Phi_n(\textbf{V}) + \omega_{n+1}(\textbf{V})\phi_{n+1}(\textbf{V})}{W_{n}(\textbf{V}) + \omega_{n+1}(\textbf{V}}\\
W_{n+1}(\textbf{V}) &= W_n(\textbf{V}) + \omega_n(\textbf{V}) \\
\end{align*}

Here, the voxel weighting is defined as:

\[
    \omega(\textbf{V}) = \left\{\begin{array}{lr}
    1, & \text{for } \phi_{true}(\textbf{V}) > - \eta\\
    0, & \text{otherwise}\\
    \end{array}\right\}
\]

$\eta$ is a hyper-parameter that determines the expected thickness of the object 

\subsection{how the fuck it works}


\section{KillingFusion}

KillingFusion operates very similarly to SDF-2-SDF, but introduces two other terms to the error function to enable it to estimate a non-rigid transformation.

\begin{align*}
E_{non-rigid}(\Psi) &= E_{data}(\Psi) + \omega_kE_{Killing}(\Psi) + \omega_sE_{level-set}(\Psi)\\
E'_{data}(\Psi) &= (\phi_n(\Psi) - \phi_{global}) \nabla_{\phi_n}(\Psi)\\
E'_{Killing}(\Psi) &= 2H_{uvw}(vec(J^{\top}_{\Psi}) vec\big(J_{\Psi})\big)\begin{pmatrix}1\\\gamma \end{pmatrix}\\
E'_{level-set}(\Psi) &= \frac{|\nabla_{\phi_n}(\Psi)| - 1}{|\nabla_{\phi_n}(\Psi)|_\epsilon}H_{\phi_n}(\Psi)\nabla_{\phi_n}(\Psi)
\end{align*}

This error function imposes three requirements on a potential solution:
\begin{itemize}
\item \textbf{Data term}: Similar to DynamicFusion, contains a component for non-rigid transformation.
\item \textbf{Killing condition}: This term requires the flow field to be a Killing vector field. If it satisifies this condition, then it will produce isometric motion Since a perfect isometric motion would impose only rigid transformations, this term only needs to be minimised so that the motion is approximately rigid. Additionally, the requirement has been weakened to make it computationally more efficient.
\item \textbf{Level set property}: The minimisation of this term ensures that the gradient of the SDF function remains approximately 1
\end{itemize}

The additional parameters $\omega_k$ and $\omega_s$ are used to adjust the relative importance of the three terms when calculating $E_{non-rigid}$.

This error function is also much more easily parallelised, leading to greatly increased performance.\\


\subsection{Killing vector field}

\subsection{Deformation field}
What it is theoretically:
Mapping psi: R3 -> R3 which describes a translation of that voxel which deforms 3D space.

How i did it:
3d voxel grid of vectors

How its estimated:
Gradient descent scheme

Define error function!

Rigid deformation done first:
Difference of distances in SDFs, phin to phiglobal

Non rigid deformation done next
Two more componenets added
Killing energy:
Preserves the smoothness of deformation field

Level set energy:
Keeps the magnitude of the gradient of the SDF approximately 1

Then we need to find the gradient of the error functions for the gradient descent schema

Rigid:

Non-rigid:

Optimisations
Designed to be done in parallel: i did multi-threaded on CPU, could also do GPU but not sure about time :/


%%%%%%%%%%%%%%%%%%%%%%%%%%%%%%%%%%%%
\chapter{Contribution}


%%%%%%%%%%%%%%%%%%%%%%%%%%%%%%%%%%%%
\chapter{Experimental Results}


%%%%%%%%%%%%%%%%%%%%%%%%%%%%%%%%%%%%
\chapter{Conclusion}


%% bibliography
\bibliographystyle{apa}


\end{document}
