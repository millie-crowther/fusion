\documentclass[12pt,twoside]{report}

% some definitions for the title page
\newcommand{\reporttitle}{asdfasdf}
\newcommand{\reportauthor}{Your name}
\newcommand{\supervisor}{Name of supervisor}
\newcommand{\reporttype}{Type of Report/Thesis}
\newcommand{\degreetype}{Type of degree} 

% load some definitions and default packages
\input{includes}

% load title page
\begin{document}
\input{titlepage}


% page numbering etc.
\pagenumbering{roman}
\clearpage{\pagestyle{empty}\cleardoublepage}
\setcounter{page}{1}
\pagestyle{fancy}

%%%%%%%%%%%%%%%%%%%%%%%%%%%%%%%%%%%%
\begin{abstract}
Your abstract.
\end{abstract}

\cleardoublepage
%%%%%%%%%%%%%%%%%%%%%%%%%%%%%%%%%%%%
\section*{Acknowledgments}
Comment this out if not needed.

\clearpage{\pagestyle{empty}\cleardoublepage}

%%%%%%%%%%%%%%%%%%%%%%%%%%%%%%%%%%%%
%--- table of contents
\fancyhead[RE,LO]{\sffamily {Table of Contents}}
\tableofcontents 


\clearpage{\pagestyle{empty}\cleardoublepage}
\pagenumbering{arabic}
\setcounter{page}{1}
\fancyhead[LE,RO]{\slshape \rightmark}
\fancyhead[LO,RE]{\slshape \leftmark}

%%%%%%%%%%%%%%%%%%%%%%%%%%%%%%%%%%%%
\chapter{Introduction}

\begin{figure}[tb]
\centering
\includegraphics[width = 0.4\hsize]{./figures/imperial}
\caption{Imperial College Logo. It's nice blue, and the font is quite stylish. But you can choose a different one if you don't like it.}
\label{fig:logo}
\end{figure}

Figure~\ref{fig:logo} is an example of a figure. 

\section{Motivation}

The reconstruction of 3-dimensional scenes using RGB-D data has been the subject of a lot of research recently. Due to the wide range of applications in fields such as Robotics and Augmented Reality, it has gained much interest.

However, most existing systems are designed only to reconstruct static geometry. This leads to hard limitations in the adaptability of the system to a changing environment. Worse still, highly deformable or dynamic objects such as a moving person cannot be reconstructed at all by a static system.

There has been a lot of interest in techniques for non-rigid reconstruction, but despite it being a hot research topic, few implementations of proposed systems are widely available.

The aim of this project is to provide such an implementation that can reconstruct a rigid object from a  non-rigid scene in real-time.


%%%%%%%%%%%%%%%%%%%%%%%%%%%%%%%%%%%%
\chapter{Background}

\section{Kinect Camera}

The first and most fundamental element of the system is the sensor used to obtain information about the surface to be reconstructed. The most commonly used sensor for this purpose is the Kinect camera. Originally designed as a commodity level camera for home games consoles, it was available directly to consumers. This means that Kinect cameras are widely and cheaply available to use.

However, they provide high quality information considering their price. This is provided to the system in the form of an ordinary RGB image, augmented with a fourth parameter indicating the perceived depth from the camera at that pixel. Using the available depth information, it is also possible to compute the normal to the surface at each pixel of the image. In some cases, depth information may be missing at certain pixels, producing so-called `holes' in the depth image. Errors such as these are not difficult to overcome but must be handled.


The sensor is capable of producing an RGB-D image to the system in 640x480 resolution at up to 30Hz. This is important, as it determines the amount of information that can be added to the system at each update. Additionally, it places a hard limit on the update rate of the system, which can never exceed the sensor rate.

\section{KillingFusion}

KillingFusion operates very similarly to DynamicFusion, but introduces a much more sophisticated error function for estimating the warp field.

\begin{align*}
E_{non-rigid}(\Psi) &= E_{data}(\Psi) + \omega_kE_{Killing}(\Psi) + \omega_sE_{level-set}(\Psi)\\
E'_{data}(\Psi) &= (\phi_n(\Psi) - \phi_{global}) \nabla_{\phi_n}(\Psi)\\
E'_{Killing}(\Psi) &= 2H_{uvw}(vec(J^{\top}_{\Psi}) vec\big(J_{\Psi})\big)\begin{pmatrix}1\\\gamma \end{pmatrix}\\
E'_{level-set}(\Psi) &= \frac{|\nabla_{\phi_n}(\Psi)| - 1}{|\nabla_{\phi_n}(\Psi)|_\epsilon}H_{\phi_n}(\Psi)\nabla_{\phi_n}(\Psi)
\end{align*}

This error function imposes three requirements on a potential solution:
\begin{itemize}
\item \textbf{Data term}: Similar to DynamicFusion, contains a component for non-rigid transformation.
\item \textbf{Killing condition}: This term requires the flow field to be a Killing vector field. If it satisifies this condition, then it will produce isometric motion Since a perfect isometric motion would impose only rigid transformations, this term only needs to be minimised so that the motion is approximately rigid. Additionally, the requirement has been weakened to make it computationally more efficient.
\item \textbf{Level set property}: The minimisation of this term ensures that the gradient of the SDF function remains approximately 1
\end{itemize}

The additional parameters $\omega_k$ and $\omega_s$ are used to adjust the relative importance of the three terms when calculating $E_{non-rigid}$.

This error function is also much more easily parallelised, leading to greatly increased performance.\\


\noindent\textbf{Limitations}

KillingFusion has been shown to handle fast inter-frame motion and topological changes much better than alternatives such as DynamicFusion or VolumeDeform. This means that it is more desirable as a solution than the other methods outlined here.

It does however still suffer from the same limitations as all of these methods in that it can only be applied to a small volume until further improvements to the method are made.


%%%%%%%%%%%%%%%%%%%%%%%%%%%%%%%%%%%%
\chapter{Contribution}


%%%%%%%%%%%%%%%%%%%%%%%%%%%%%%%%%%%%
\chapter{Experimental Results}


%%%%%%%%%%%%%%%%%%%%%%%%%%%%%%%%%%%%
\chapter{Conclusion}


%% bibliography
\bibliographystyle{apa}


\end{document}
